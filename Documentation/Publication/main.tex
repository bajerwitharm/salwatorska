% Publication name: Inexpensive system for WiFi network access and monitoring
% Author: Marcin Bajer
% Institution: ABB Corporate Research, Krakow, Poland
\documentclass{llncs}
%
\usepackage{makeidx}  % allows for indexgeneration

\usepackage[pdftex]{graphicx}
\usepackage{graphviz}
%

\begin{document}

%
\frontmatter          % for the preliminaries
%
\pagestyle{headings}  % switches on printing of running heads
%
\mainmatter              % start of the contributions
%
\title{Inexpensive system for WiFi network access and monitoring}

\author{Marcin Bajer\inst{1} \and Anna Bajer\inst{2}}
%
\authorrunning{Marcin Bajer} % abbreviated author list (for
% running head)
%
%%%% list of authors for the TOC (use if author list has to be modified)
\tocauthor{Marcin Bajer}
%
\institute{ABB Corporate Research Krakow, Starowiślna 13A, Krakow, Poland,\\
\email{marcin.bajer@pl.abb.com}
\and
Uniwersity of Economics, Krakow, Poland, \email{ann.bajer@gmail.com}}

\maketitle              % typeset the title of the contribution

\begin{abstract}
The goal of this publication is to describe cost effective approach for creating WiFi network access 
and monitoring system in small and medium size buildings. The basic assumption that the system should 
support per user authentication access control and network usage statistics. 
In contrast to current market available solutions the presented one requires very little modification of off-the-self 
low cost routers to enable wide range of new functionalities. The core or the
system is based on OpenWRT Linux running on multiple WiFi access points spread throughout the building. 
For data storage and visualization centralized solution 
based on node.js server running on RaspberryPi development board is used.
\keywords{netowork monitoring, access control, RaspberryPi, node.js, smart home}
\end{abstract}
%
\section{Introduction}
%

It is expected that the WiFi market will continue to grow tremendously over the
next few years. One of main driving factors for this is the increasing number of
smart devices (mobiles phones, tablets, smartwatches and smart TVs\ldots). The
observed pattern is to maximum data usage by the end users when they are
stationary connected to WiFi in homes, restaurants and offices.
Additionaly, eventhough the falling prices of mobile transfer, it is common that
smart devices are configured to trigger data intensive applications such as
updates and synchronization only while being connected to WiFi.

Small businesses are often in a difficult position when it comes to WiFi
security. On one hand the growing demand for WiFi access is pusshing them
towards allowing more users to access their WiFi network, on the other hand they
demand cheap and secure solution for doing so. The main requiremnt for them is
to enable per user access and monitoring. 

Further in this paper the approach for praparation of internal WiFi network in
medium size rent house is presented. The goal of the project is not only to
provide WiFi netowrk to the users, but also to setup backbone for planned
smart home installation.

%% Start of dot diagram
\begin{figure}
\includedot[width=\textwidth]{./Graphics/NetworkTopology/topology}
\vspace{-15pt}
\caption{Topology of prepared network}
\end{figure}
%% End of dot diagram

%% Start of dot diagram
\begin{figure}
\includedot[width=\textwidth]{./Graphics/Database/schema}
\vspace{-15pt}
\caption{Topology of prepared network}
\end{figure}
%% End of dot diagram

%
% ---- Bibliography ----
%
\begin{thebibliography}{}
%
\bibitem{glob}
Global Wi-Fi Market [Indoor Wi-Fi, Outdoor Wi-Fi, Transportation Wi-Fi]: Global
Advancements, Business Models, Market Forecasts \& Analysis (2014 - 2019), 
marketsandmarkets.com,
August 2014
\end{thebibliography}

\clearpage
\addtocmark[2]{Author Index} % additional numbered TOC entry
\renewcommand{\indexname}{Author Index}
\printindex
\clearpage
\addtocmark[2]{Subject Index} % additional numbered TOC entry
\markboth{Subject Index}{Subject Index}
\renewcommand{\indexname}{Subject Index}
\input{subjidx.ind}
\end{document}